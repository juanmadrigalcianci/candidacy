% 4.	Planification future (plan de travail) / Future planning (scheme of work): ½ page
\begin{itemize}
\item There are some pictures here to be included soon \color{black}\\
\end{itemize}
\noindent There are different research directions and projects that would be nice to explore in the future, both from a computational and from a theoretical perspective. We here present an outline of the ideas that we would like to explore during the duration of this project. \\ \\ \noindent 
\begin{itemize}
	\item There are still many different ways of proposing MLMCMC algorithms, two of which have been discussed and will be studied in the near future. The first one is to extend the same framework of the parallel tempering algorithm, where multiple chains using different noise levels ( so called ``temperatures'') are run in parallel and set to exchange states using a Metropolis-Hastings acceptance-rejection step every so often, to a multi-level or even multi-index setting, where there would be temperatures and discretization levels to take into account. A similar idea to this has been implemented in the context of sequential Monte Carlo by \cite{latz2017multilevel}. 
	\item The other interesting research direction arises when combining multi-level ideas with transport maps. Transport maps have been studied in \cite{marzouk2016introduction,parno2016multiscale,parno2018transport} and used as an efficient way of accelerating  MCMC.\\ 
	\item Concerning the inversion problem, it would be interesting to move the project into more challenging problems, such as those involving Cartesian three dimensional models, or source inversion at a global scale. It is also important to apply the methods explored herein to inversion of slip faults, rather than ``just" point sources, as these represent more realistic models. 
	\item Concerning the inversion problem, it would be interesting to move the project into more challenging problems, such as those involving Cartesian three dimensional models, or source inversion at a global scale. It is also important to apply the methods explored herein to inversion of slip faults, rather than ``just" point sources, as these represent more realistic models. 
	\item We are interested in performing Bayesian inversion for the source location. However, in many cases the material properties of the earth are deemed to be uncertain, and as such can be treated as nuisance parameters in the Bayesian inversion.\color{red} this transition here can be done more smoothly \color{black} In more technical terms,  denoting by $\te^s, \te^e$ the  (unknown) source and material properties respectively, in such a way that $\te=(\te^s,\te^e)$, we would like to obtain $\pi(\te^s|y)$ based on $\pi(\te^s,\te^e|y)$. Thus, \begin{align}
\pi(\te^s|y)&=\int_{\Theta^e}\pi(\te^s,\te^e)\pi(\te^e)d\te^e \label{marg}\\ &\approx\frac{1}{N}\sum^N_{i=1}\pi(\te^s,\te^e_i), \quad \te^e\sim \pi(\te^e).\label{mcmarg}
	\end{align} We can therefore combine two ideas to obtain samples from $\pi(\te^s|y)$; we can use the pseudo-marginal MCMC  \cite{andrieu2009pseudo} to sample from  $\pi(\te^s|y)$ where we can only access $\pi(\te^s,\te^e|y)$, and, additionally, we can use a  multi-level Monte Carlo for a fast computation of the expected value (\ref{mcmarg}).  
	\item Lastly, it would be a very interesting idea to collect all these sampling strategies, multi-level or not, and implement a library  (either in MATLAB or python) that can be used as a ``blackbox'', so that other research groups can use the methods developed in this work. \\
\end{itemize}
